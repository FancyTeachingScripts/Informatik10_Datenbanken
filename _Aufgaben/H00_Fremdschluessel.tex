\Hefteintrag{1.8}{Tabellenbeziehungen: Fremdschlüssel}{ \Large
Wenn Datensätze mittels Primärschlüssel in einer anderen Tabelle verwendet werden, spricht man dort von einem Fremdschlüssel. Im Tabellenschema werden die \LoesungLuecke{Fremdschlüssel}{8cm} durch (\hspace{8cm}) (manchmal auch \hspace{8cm}) markiert. Ein Beispiel in SQL-Island ist der Häuptling eines Dorfes, der in der Tabelle Dorf mittels bewohnernr eingetragen wird. Die \emphBlue{bewohnernr} ist hierbei \emphBlue{\LoesungLuecke{}{8cm}} in der \emphBlue{Tabelle Bewohner} und \emphGreen{\LoesungLuecke{}{8cm}} in der \emphGreen{Tabelle Dorf} (heißt hier aber \emphGreen{haeuptling}).

}

