\Hefteintrag{1.8}{Tabellenbeziehungen: Fremdschlüssel}{ \Large
Wenn Datensätze mittels Primärschlüssel in einer anderen Tabelle verwendet werden, spricht man dort von einem Fremdschlüssel. Im Tabellenschema werden die \LoesungLuecke{Fremdschlüssel}{8cm} durch (\hspace{8cm}) (manchmal auch \hspace{8cm}) markiert. Ein Beispiel in SQL-Island ist der Häuptling eines Dorfes, der in der Tabelle Dorf mittels bewohnernr eingetragen wird. Die \emphColB{bewohnernr} ist hierbei \emphColB{\LoesungLuecke{}{8cm}} in der \emphColB{Tabelle Bewohner} und \emphColC{\LoesungLuecke{}{8cm}} in der \emphColC{Tabelle Dorf} (heißt hier aber \emphColC{haeuptling}).

}

