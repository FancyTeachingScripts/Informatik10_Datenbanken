\Aufgabe{\ifbeamer\includegraphics[height=15pt]{_Aufgaben/img/artemis.png}~~~~\fi Wdh: SQL Basics}{
    Bearbeite die Aufgabe \emphColB{Wdh - SQL Basics} auf \url{artemis.tum.de}. Artemis gibt dir immer, wenn du auf Submit drückst, die ersten Zeilen der Ergebnistabelle und ob deine SQL-Abfrage (bzw. welche Teile von ihr) richtig sind, aus.
    
    Wenn du eine Abfrage richtig hast, notiere sie unten im Skript.
    
    Falls du bei Gruppierung und Aggregatfunktionen Schwierigkeiten hast, hilft dir dieses \emphColC{Video (bitte Kopfhörer verwenden!)}: \UrlAndCode{bycs.link/simpleclub-group-sort-aggregat}
    
    \setcounter{tmp}{1}
    
   \vspace{0.3cm} \emphColB{1)} 
   Vervollständige die SQL-Abfrage so, dass sie ID, Name, Art und URL aller Freibäder ausgibt.
    \\\LoesungKaro{SELECT id, name, art, url\\FROM Schwimmbad\\WHERE art='Freibad' }{6}
}
\UnterAufgabe{Wdh: SQL Basics}{
   \vspace{0.3cm} \emphColB{2)} 
   Schreibe eine SQL-Abfrage, die ausgibt, wie viele Gemeinden es im Regierungsbezirk 'Oberbayern' gibt. 
    \\\LoesungKaro{SELECT COUNT(*)\\FROM Gemeinde\\WHERE regierungsbezirk='Oberbayern'}{6}
   
   \vspace{0.3cm} \emphColB{3)} 
   Schreibe eine SQL-Abfrage, die Name, Straße und URL (also die Internetadresse) alle Zoos in der Gemeinde mit Schluessel '09162000' ausgibt.
    \\\LoesungKaro{SELECT name, strasse, url\\FROM Zoo\\WHERE gemeindeschluessel = '09162000'}{6}
}
   
\UnterAufgabe{Wdh: SQL Basics}{   
   \vspace{0.3cm} \emphColB{4)} 
   Schreibe eine SQL-Abfrage, die die Summe aller weiblichen Einwohnerinnen und die Summe aller männlichen Einwohner gruppiert nach Regierungsbezirk und den Namen des jeweiligen Regierungsbezirks ausgibt.
    \\\LoesungKaro{SELECT regierungsbezirk, SUM(einwohner\_w), SUM(einwohner\_m)\\FROM gemeinde\\GROUP BY regierungsbezirk}{6}
   
   \vspace{0.3cm} \emphColB{5)} 
   Schreibe eine SQL-Abfrage, die die durchschnittliche Fläche der Gemeinde eines Kreises (=Landkreis) und den Namen und Regierungsbezirk des jeweiligen Landkreises anzeigt. Sortiere die Ausgabe nach Name des Landkreises.
   \hinweis{Achtung: Du kannst bei der Verwendung von Gruppierung nur Spalten, nach denen gruppiert wird, und solche, die mit Aggregatfunktionen zusammengefasst werden, anzeigen! Überlege, wie du dieses Problem hier lösen kannst.}
    \\\LoesungKaro{SELECT regierungsbezirk, kreis, avg(flaeche)\\FROM Gemeinde\\GROUP BY regierungsbezirk,kreis\\ORDER BY kreis}{6}
}
   
\UnterAufgabe{Wdh: SQL Basics}{   
   \vspace{0.3cm} \emphColB{6)} 
   Schreibe eine SQL-Abfrage, die die Namen und Einwohnerzahlen aller Gemeinde, die mehr als 100.000 männliche und mehr als 100.000 weibliche Einwohner:innen haben, ausgibt.
    \\\LoesungKaro{SELECT name, einwohner\_m, einwohner\_w\\FROM Gemeinde\\WHERE einwohner\_m > 100000\\AND einwohner\_w > 100000}{6}
   
   \vspace{0.3cm} \emphColB{7)} 
   Schreibe eine SQL-Abfrage, die die Namen und Einwohnerzahlen aller Gemeinde, die mehr als 75.000 männliche oder mehr als 75.000 weibliche Einwohner:innen haben, ausgibt.
    \\\LoesungKaro{SELECT name, einwohner\_m, einwohner\_w \\FROM Gemeinde\\WHERE einwohner\_m > 75000\\OR einwohner\_w > 75000}{6}
}
   
\UnterAufgabe{Wdh: SQL Basics}{      
   \vspace{0.3cm} \emphColB{8)} 
   Schreibe eine SQL-Abfrage, die Name, Landkreis, Fläche und die Einwohnerzahlen aller Gemeinden ausgibt, die jeweils mehr als 50.000 männliche und weibliche Einwohner:innen oder eine Fläche größer als 100 km² hat.
    \\\LoesungKaro{SELECT name, kreis, flaeche, einwohner\_m, einwohner\_w\\FROM Gemeinde\\WHERE (einwohner\_m > 50000 AND einwohner\_w > 50000)\\OR flaeche > 100}{6}
   
   \vspace{0.3cm} \emphColB{9)} 
   Schreibe eine SQL-Abfrage, die die durchschnittlichen männlichen und weiblichen Einwohnerzahlen aller Gemeinde mit mehr als 100 km² Fläche pro Landkreis und den Namen des jeweiligen Landkreises ausgibt.
    \\\LoesungKaro{SELECT kreis, AVG(einwohner\_m), AVG(einwohner\_w)\\FROM Gemeinde\\WHERE flaeche > 100\\GROUP BY kreis}{6}
}
   
\UnterAufgabe{Wdh: SQL Basics}{      
   \vspace{0.3cm} \emphColB{10)} 
   Schreibe eine SQL-Abfrage, die die Anzahl von Wanderwegen, die zu einer Gemeinde führen in einer Spalte Anzahl und den jeweiligen Gemeindeschlüssel absteigend nach Anzahl sortiert, ausgibt.
    \\\LoesungKaro{SELECT gemeindeschluessel,COUNT(*) as Anzahl\\FROM Wanderweg\_zu\_Gemeinde\\GROUP BY gemeindeschluessel\\ORDER BY Anzahl DESC}{6}
}
