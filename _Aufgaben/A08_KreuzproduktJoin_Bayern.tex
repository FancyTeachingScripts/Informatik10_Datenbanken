\Aufgabe[45]{\ifbeamer\includegraphics[height=15pt]{_Aufgaben/img/artemis.png}~~~~\fi SQL mit Kreuzprodukt und Join}{

Bearbeite diese Aufgabe auf \url{artemis.tum.de}. Du bekommst eine automatische Rückmeldung, ob deine Abgabe korrekt ist.
Alle Aufgaben beziehen sich auf die Datenbank mit untem stehendem Klassendiagramm. Eine Online-Version gibt es unter \url{www.dbiu.de/bayern/}, dort ist auch das Tabellenschema zu finden.

Gib immer genau die geforderten Daten aus und nicht mehr. Sortiere nicht, wenn du nicht dazu aufgefordert wirst.

\emphColB{Notiere unten anschließend deine korrekten SQL-Abfragen unten.}

}

\UnterAufgabe{\ifbeamer\includegraphics[height=15pt]{_Aufgaben/img/artemis.png}~~~~\fi SQL mit Kreuzprodukt und Join}{
% TODO: Ersetzen
\begin{center}
% \ifbeamer
%     \includegraphics[width=0.6\textwidth]{img/Bayern_DB.png}
% \else
%     \includegraphics[width=\textwidth]{img/Bayern_DB.png}
% \fi


\begin{tikzpicture}
    \begin{class}{Gemeinde}{0,0}
        \attribute{String schluessel}
        \attribute{String regierungsbezirk}
        \attribute{String kreis}
        \attribute{String name}
        \attribute{String zusatz}
        \attribute{String plz}
        \attribute{float flaeche}
        \attribute{int einwohner$\_$m}
        \attribute{int einwohner$\_$w}
    \end{class}
    \begin{class}{Freizeitpark}{-6.5,3}
        \attribute{int id}
        \attribute{String name}
        \attribute{String strasse}
        \attribute{String url}
        \attribute{float breitengrad}
        \attribute{float laengengrad}
    \end{class}
    \begin{class}{Schwimmbad}{6.5,3}
        \attribute{int id}
        \attribute{String name}
        \attribute{String art}
        \attribute{String strasse}
        \attribute{String url}
        \attribute{float breitengrad}
        \attribute{float laengengrad}
    \end{class}
    \begin{class}{Radweg}{0,3}
        \attribute{String radweg$\_$id}
        \attribute{String name}
    \end{class}
    \begin{class}{Wanderweg}{6.5,-4.2}
        \attribute{String wanderweg$\_$id}
        \attribute{String name}
    \end{class}
    \begin{class}{Zoo}{-6.5,-2.5}
        \attribute{int id}
        \attribute{String name}
        \attribute{String strasse}
        \attribute{String url}
        \attribute{float breitengrad}
        \attribute{float laengengrad}
    \end{class}
    
    \uniAssociationAngle
    {Freizeitpark.south}{270}
    {left}{n}
    {gemeindeschluessel}
    {1}{below}
    {180}{[shift={(0,1.5)}]Gemeinde.west}
    
    \uniAssociationAngle
    {Schwimmbad.south}{270}
    {right}{n}
    {gemeindeschluessel}
    {1}{below}
    {0}{[shift={(0,1.3)}]Gemeinde.east}
    
    
    \uniAssociationAngle
    {[shift={(0,0)}]Zoo.south east}{0}
    {right}{n}
    {gemeindeschluessel}
    {1}{below}
    {180}{[shift={(0,0)}]Gemeinde.south west}
    
    \biAssociationAngle
    {[shift={(2.5,0)}]Radweg.south}{270}
    {left}{n}
    {Radweg$\_$zu$\_$Gemeinde}
    {m}{left}
    {90}{[shift={(2,0)}]Gemeinde.north}
    
    \biAssociationAngle
    {[shift={(-1,0)}]Wanderweg.north}{90}
    {right}{n}
    {Wanderweg$\_$zu$\_$Gemeinde}
    {m}{above}
    {0}{[shift={(0,-1)}]Gemeinde.east}
    
    \biAssociationAngle
    {[shift={(1.7,0)}]Gemeinde.south}{270}
    {left}{n}
    {Nachbargemeinde}
    {m}{right}
    {270}{[shift={(-0.1,0)}]Gemeinde.south east}
\end{tikzpicture}


\end{center}
}

\UnterAufgabe{\ifbeamer\includegraphics[height=15pt]{_Aufgaben/img/artemis.png}~~~~\fi SQL mit Kreuzprodukt und Join}{
Verändere die SQL-Abfrage so, dass die Namen und Internetadressen (=url) aller Zoos und der Name und Regierungsbezirk der jeweiligen Gemeinde ausgegeben wird:

\vspace{0.3cm}

\large
\emphColB{SELECT Zoo.name, Gemeinde.name} \LoesungLuecke{,Gemeinde.regierungsbezirk, Zoo.url}{8cm}

\vspace{0.5cm}
\emphColB{FROM Zoo, Gemeinde}

\vspace{0.5cm}
\LoesungLuecke{WHERE Zoo.gemeindeschluessel = Gemeinde.schluessel}{17cm}

}

\UnterAufgabe{\ifbeamer\includegraphics[height=15pt]{_Aufgaben/img/artemis.png}~~~~\fi SQL mit Kreuzprodukt und Join}{

Verändere die SQL-Abfrage so, dass die Namen und Straßen aller Freizeitparks und die Namen der jeweils zugehörigen Gemeinde ausgegeben wird.

\vspace{0.3cm}
\large
\emphColB{SELECT Freizeitpark.name, Gemeinde.name} \LoesungLuecke{, Freizeitpark.strasse}{8cm}

\vspace{0.5cm}
\emphColB{FROM Freizeitpark, Gemeinde}

\vspace{0.5cm}
\LoesungLuecke{WHERE Gemeinde.schluessel = Freizeitpark.gemeindeschluessel}{17cm}

}

\UnterAufgabe{\ifbeamer\includegraphics[height=15pt]{_Aufgaben/img/artemis.png}~~~~\fi SQL mit Kreuzprodukt und Join}{
Schreibe eine SQL-Abfrage, die Namen und Art aller Schwimmbäder und den Namen und alle Einwohnerzahlen der zugehörigen Gemeinden ausgibt.

\LoesungKaro{SELECT Schwimmbad.name, Schwimmbad.art, \\ Gemeinde.name, Gemeinde.einwohner$\_$m, Gemeinde.einwohner$\_$w\\
FROM Schwimmbad, Gemeinde\\
WHERE Gemeinde.schluessel = Schwimmbad.gemeindeschluessel}{8}
}

\UnterAufgabe{\ifbeamer\includegraphics[height=15pt]{_Aufgaben/img/artemis.png}~~~~\fi SQL mit Kreuzprodukt und Join}{

Schreibe eine SQL-Abfrage, die die Anzahl an Schwimmbädern in Gemeinden mit \emph{mehr} als 1000 weiblichen Einwohnerinnen ausgibt.

\emph{Tipp: Hier brauchst du mehrere verknüpfte Bedingungen}

\LoesungKaro{SELECT COUNT(*)\\
FROM Schwimmbad, Gemeinde\\
WHERE Gemeinde.schluessel = Schwimmbad.gemeindeschluessel\\
  AND Gemeinde.einwohner$\_$w > 1000}{8}
}

\UnterAufgabe{\ifbeamer\includegraphics[height=15pt]{_Aufgaben/img/artemis.png}~~~~\fi SQL mit Kreuzprodukt und Join}{
Schreibe eine SQL-Abfrage, die die Namen aller Gemeinde in Oberbayern oder Niederbayern, zu denen ein Wanderweg führt, ausgibt. Dopplungen dürfen auftreten und sollte nicht entfernt werden!

\emph{Tipp: Hier brauchst du wieder mehrere verknüpfte Bedingungen. Überlege bei der Verknüpfung von Bedingungen, ob du Klammern setzen musst!}

\LoesungKaro{SELECT Gemeinde.name\\
FROM Gemeinde,Wanderweg$\_$zu$\_$Gemeinde\\
WHERE Gemeinde.schluessel = Wanderweg$\_$zu$\_$Gemeinde.gemeindeschluessel\\
AND (Gemeinde.regierungsbezirk='Oberbayern' \\
OR Gemeinde.regierungsbezirk='Niederbayern')}{10}
}

\UnterAufgabe{\ifbeamer\includegraphics[height=15pt]{_Aufgaben/img/artemis.png}~~~~\fi SQL mit Kreuzprodukt und Join}{
Schreibe eine SQL-Abfrage, die aus den Tabellen Gemeinde und Wanderweg$\_$zu$\_$Gemeinde die Anzahl der Wanderwege, die zu Gemeinden mit mehr als 500 000 männlichen Einwohnern führen, ausgibt.


\LoesungKaro{SELECT COUNT(*)\\
FROM Gemeinde, Wanderweg$\_$zu$\_$Gemeinde\\
WHERE Gemeinde.schluessel = Wanderweg$\_$zu$\_$Gemeinde.gemeindeschluessel\\
  AND einwohner$\_$m > 500000}{8}
}

\UnterAufgabe{\ifbeamer\includegraphics[height=15pt]{_Aufgaben/img/artemis.png}~~~~\fi SQL mit Kreuzprodukt und Join}{
Schreibe eine SQL-Abfrage, die eine Liste mit den Namen aller Gemeinden, die ein 'Freibad' haben, und die Namen der jeweiligen Freibäder ausgibt. 

\LoesungKaro{SELECT Gemeinde.name, Schwimmbad.name\\
FROM Gemeinde, Schwimmbad\\
WHERE Gemeinde.schluessel=Schwimmbad.gemeindeschluessel\\
AND Schwimmbad.art='Freibad'}{8}
}

\UnterAufgabe{\ifbeamer\includegraphics[height=15pt]{_Aufgaben/img/artemis.png}~~~~\fi SQL mit Kreuzprodukt und Join}{
Schreibe eine SQL-Abfrage, die die Anzahl an Radwegen, die an Gemeinden im PLZ-Bereich \emphColA{größer} als 96400 angrenzen, ausgibt.

\LoesungKaro{SELECT COUNT(*)\\
FROM Gemeinde, Radweg$\_$zu$\_$Gemeinde\\
WHERE Gemeinde.schluessel=Radweg$\_$zu$\_$Gemeinde.gemeindeschluessel\\
  AND Gemeinde.plz > 96400}{8}

}

\UnterAufgabe{\ifbeamer\includegraphics[height=15pt]{_Aufgaben/img/artemis.png}~~~~\fi SQL mit Kreuzprodukt und Join}{
\begin{minipage}[t]{\textwidth}
Schreibe eine SQL-Abfrage, die die Namen aller Zoos in einer Gemeinde namens 'Erlangen' ausgibt.

\LoesungKaro{SELECT Zoo.name\\
FROM Zoo,Gemeinde\\
WHERE Zoo.gemeindeschluessel = Gemeinde.schluessel\\
AND Gemeinde.name='Erlangen'}{8}
\end{minipage}


}

\UnterAufgabe{\ifbeamer\includegraphics[height=15pt]{_Aufgaben/img/artemis.png}~~~~\fi SQL mit Kreuzprodukt und Join}{



Schreibe eine SQL-Abfrage, die die IDs aller Radwege, die zu Gemeinden in Oberfranken oder Unterfranken führen, ausgibt. Dopplungen sollen nicht entfernt werden.

\LoesungKaro{SELECT Radweg$\_$zu$\_$Gemeinde.radweg$\_$id\\
FROM Radweg$\_$zu$\_$Gemeinde, Gemeinde\\
WHERE Gemeinde.schluessel = Radweg$\_$zu$\_$Gemeinde.gemeindeschluessel\\
  AND (Gemeinde.regierungsbezirk = 'Oberfranken' \\
  OR Gemeinde.regierungsbezirk='Unterfranken')}{12}
}