

\newcommand{\dbprojekt}[9]
{
\Datum{#1 bis #9}

\section{Rahmenbedingungen}
\begin{itemize}
    \item Bearbeitung in Teams bestehend aus 4 Personen
    \item Melden der Gruppen bis spätestens vor der ersten Deadline bei Herr Herrmann.
    \item Entwicklung nach dem Wasserfall-Modell:  \url{bycs.link/inf10-wasserfall} oder \\\url{de.wikipedia.org/wiki/Wasserfallmodell}
    \item Als Datenbanksoftware wird YoungDB verwendet (kostenlos verfügbar unter: \url{klassenkarte.de/index.php/tools/youngdb/}, \emph{Hinweis: Es sind nur SELECT, keine INSERT oder UPDATE Befehle möglich.}).
    \item Zusätzlich zur Arbeit im Informatik-Unterricht wird auch \emphBlue{Arbeit zuhause} notwendig sein. \emphBlue{Arbeitsteilung ist erwünscht!}
    \item Empfehlung: Dateien auf BYCS-Drive speichern.
    \item Alle Abgaben im PDF-, YKD- (Young DB Datenbankmodell) oder YDB- (Young DB Datenbank) Format.
     (per BYCS-Drive oder Mebis, wird rechtzeitig bekannt gegeben).
\end{itemize}

\section{Inhaltliche Anforderungen}
\begin{itemize}
    \item Datenbank mit mindestens 4 inhaltlichen Tabellen (Beziehungstabellen zusätzlich)
    \item alle 3 Kardinalitäts-Typen
    \item Mindestens: Erfüllung der 1. und 2. Normalform
    \item Bonus-Punkte: Erfüllung der 3. Normalform
    \item Auslesen von Daten (zur Erfüllung der Anforderungen) ausschließlich über SQL-Abfragen.
\end{itemize}

\section{Grober Zeitplan}

\begin{enumerate}
    \item \textbf{Woche ab #1:} Anforderungsanalyse und Einarbeitung in das Wasserfallmodell\\
       $\rightarrow$  \emphBlue{Abgabe: Lastenheft bis #2, 23:59 Uhr}
    \item \textbf{Woche ab #2:} Datenbankentwurf\\
       $\rightarrow$  \emphBlue{Abgabe: Pflichtenheft bis #3, 23:59 Uhr}
    \item \textbf{Woche ab #3:} Umsetzung der Datenbank
    \item \textbf{Woche ab #4:} Umsetzung der Datenbank
    \item \textbf{Woche ab #5:} Umsetzung der Datenbank\\
       $\rightarrow$  \emphBlue{Abgabe: Datenbank bis #6, #7}
    \item \textbf{Woche ab #6:} Tests + Dokumentation
    \item \textbf{Woche ab #8:} Abschluss der Dokumentation und Reflexion\\
       $\rightarrow$  \emphBlue{Abgabe: Dokumentation und Reflexion bis #9, 23:59 Uhr}
\end{enumerate}

\section{Anforderung einzelner Phasen}

\subsection{Anforderungsanalyse}
\begin{itemize}
    \item Einarbeitung in Wasserfallmodell und User Stories etc. + (parallel) Entwurf der Projektidee
    \item Allgemeine Beschreibung des Produkts und Einsatz (\emphBlue{Lastenheft})
    \item 5 funktionale und 3 nicht-funktionale Anforderungen (\emphBlue{Lastenheft})
    \item mindestens 2 User Stories (\emphBlue{Lastenheft})
    \item Empfehlung: Formulierung einzelner TODOs auf einem Project Board.
\end{itemize}

\subsection{Datenbankentwurf}
\begin{itemize}
    \item Entwurf des Datenbankschemas mit \url{apollon.ase.in.tum.de}
    $\rightarrow$ \emphOrange{Klassendiagramm} (\emphBlue{Pflichtenheft})
    \item Entwurf User Interface (= Ausgabetabellen der SQL-Abfragen)
    $\rightarrow$ Festhalten mittels Skizzen/Beschreibung (\emphBlue{Pflichtenheft})
    \item Zuordnung der Inhalt zu den Anforderung aus dem Lastenheft (\emphBlue{Pflichtenheft}).
\end{itemize}

\subsection{Umsetzung}
\begin{itemize}
    \item Umsetzung des Plans aus dem Pflichtenheft
    \item Haltet euch möglichst genau an euer Pflichtenheft. Änderungen sollen nur notfalls vorgenommen werden und müssen begründet werden.
    \item Bearbeitung möglichst arbeitsteilig.
    \item \emphOrange{Achtung: } Nach einer Änderung des Datenbankmodells in YoungDB muss eine neue Datenbank generiert und alle Datensätze neu eingetragen werden! 
\end{itemize}

\subsection{Tests}
\begin{itemize}
    \item Kontinuierliches Testen während der Umsetzung.
    \item Überprüfung des Endprodukts zuerst anhand des \emphOrange{Pflichtenhefts} anschließend anhand des \emphOrange{Lastenhefts}
    \item Gesamtbeurteilung, ob das Projekt erfolgreich umgesetzt wurde.
    \item Festhalten aller Ergebnisse in einem Test-Protokoll (eine Auflistung, welche Anforderung wann von wem geprüft wurde und ob die Überprüfung erfolgreich war und wenn nicht, wieso).
\end{itemize}

\subsection{Dokumentation}
\begin{itemize}
    \item Inhalt: \emphBlue{Lastenheft, Pflichtenheft, Testprotokoll, Bedienungsanleitung, Finales Datenbankschema} (mit allen vorgenommenen Änderungen), \emphBlue{alle SQL-Abfragen} (SQL-Code)
    \item Optisch ansprechendes Dokument!
\end{itemize}

\subsection{Bericht/Reflexion über...}
\begin{itemize}
    \item ...die einzelnen Projektphasen und aufgetretene Schwierigkeiten.
    \item ...während der Umsetzung erfolgte Änderungen gegenüber Pflichtenheft.
    \item ...was ihr wieder so/anders machen würdet.
    \item ...die erfolgte Arbeitsteilung.
\end{itemize}

\section{Benotung}
\subsection{Note 1: Produkt und Arbeitsphase}
\begin{itemize}
    \item Funktionalität der Datenbank, Erfüllung der Spezifikation (Anforderungen)
    \item Umsetzung des Wasserfallmodells
    \item Beobachtung während Arbeitsphase
\end{itemize}

\subsection{Note 2: Dokumentation und Reflexion}
\begin{itemize}
    \item Verständlichkeit, Sprache
    \item Übereinstimmung mit dem Produkt
    \item differenzierte Reflexion der eigenen Arbeit
\end{itemize}
}