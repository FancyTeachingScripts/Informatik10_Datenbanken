\Hefteintrag{1.9}{SQL-Abfragen mit Join bei m:n-Beziehungen}{
Um zwei Tabellen, die eine m:n-Beziehung miteinander haben, zu joinen (also ihren Join zu bilden und in der Ergebnistabelle nur \LoesungLuecke{zusammengehörende}{7cm} Datensätze zu haben), muss man:
\begin{itemize}
    \item Daten aus allen \LoesungLuecke{drei}{2cm} Tabellen abfragen (also diese nach \emphColA{\LoesungLuecke{FROM}{3cm}} auflisten).
    \item Die \emphColA{\LoesungLuecke{Beziehungstabelle}{8cm}} einzeln mit den normalen Tabellen joinen. Hierfür benötigt man \emphColA{\LoesungLuecke{zwei}{2cm}} Join-Bedingungen, die mit \emphColA{\LoesungLuecke{AND}{2cm}} verknüpft werden.
\end{itemize}

\emphColA{Beispiel:}

\emphColC{SELECT} Lehrkraft.*, Schulklasse.*\\
\emphColC{FROM} Lehrkraft, Schulklasse, \emphColB{\LoesungLuecke{Lehrer\_unterricht\_Klasse}{10cm}}\\
\emphColC{WHERE} \emphColB{Lehrer\_unterricht\_Klasse.lehrer} = \LoesungLuecke{Lehrkraft.kuerzel}{8cm} \\
\emphColC{AND}  \emphColB{Lehrer\_unterricht\_Klasse.klasse} = \LoesungLuecke{Schulklasse.bezeichner}{8cm}

}
