\Hefteintrag{1.8}{Tabellenbeziehungen: Fremdschlüssel}{
    Wenn Datensätze mittels Primärschlüssel in einer anderen Tabelle verwendet werden, spricht man dort von einem Fremdschlüssel. Im \emphColA{Tabellenschema} werden die \emphColA{\LoesungLuecke{Fremdschlüssel}{8cm}} durch \emphColA{$\overline{\textbf{\LoesungLeer{"uberstreichen}{8cm}}}$} (manchmal auch \emphColA{\dotuline{\LoesungLeer{unterpunkten}{8cm}}}) markiert. Ein Beispiel in SQL-Island ist der Häuptling eines Dorfes, der in der Tabelle Dorf mittels bewohnernr eingetragen wird. Die \emphColB{bewohnernr} ist hierbei \emphColB{\LoesungLuecke{Primärschlüssel}{8cm}} in der \emphColB{Tabelle Bewohner} und \emphColC{\LoesungLuecke{Fremdschlüssel}{8cm}} in der \emphColC{Tabelle Dorf} (heißt hier aber \emphColC{haeuptling}).

}

