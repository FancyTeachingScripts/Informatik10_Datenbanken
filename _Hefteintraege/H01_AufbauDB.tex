\Hefteintrag{1.5}{Wdh: Aufbau von (relationalen) Datenbanken}{

    Datenbanken speichern Datensätze in \emphColA{\LoesungLuecke{Tabellen}{5cm}}. Die \emphColB{\LoesungLuecke{Spaltenüberschriften}{7cm}} repräsentieren die \emphColB{Attribute} (Synonym: Feld) und bilden zusammen eine \emphColor{yellow}{Klasse}. Die \emphColC{\LoesungLuecke{Datensätze}{5cm} (=Zeilen)} entsprechen \emphColC{Objekten} und in den Spalten stehen die Attributwerte.
    Jede Tabelle hat einen \emphColor{red}{\LoesungLuecke{Primärschlüssel}{6cm} (oft auch „ID“)}, der Datensätze eindeutig identifiziert. Oft werden die Datensätze hiermit einfach durchnummeriert. Im Tabellenschema wird er unterstrichen und im Klassendiagramm immer als erstes Attribut aufgelistet.
    
    Der Aufbau einer Tabelle kann mit  \emphColA{\LoesungLuecke{Klassenkarte}{5cm}} oder \emphColA{\LoesungLuecke{Tabellenschema}{5cm}} dargestellt werden. Dessen Aufbau ist:

    {
        \fontfamily{pcr}\selectfont
        \textbf{TABELLENNAME(\underline{Datentyp Primärschlüssel} , Datentyp Spalte1, Datentyp Spalte2, …)}
    }

        Zum Beispiel:

        \LoesungLine{Person(\underline{int id}, String name, int alter, …)}{1}
    
}