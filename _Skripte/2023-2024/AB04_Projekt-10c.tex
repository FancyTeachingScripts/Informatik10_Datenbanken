\Testart{Projektarbeit}
\Titel{Datenbanken}

\Datum{Abgabe: 09.02.2024, 23:59 Uhr}

\section{Rahmenbedingungen}
\begin{itemize}
    \item Bearbeitung in Teams bestehend aus 4 Personen
    \item Melden der Gruppen bis spätestens vor der ersten Deadline bei Herrmann, damit Mebis freigeschaltet werden kann.
    \item Entwicklung nach dem Wasserfall-Modell: \url{de.wikipedia.org/wiki/Wasserfallmodell}
    \item Als Datenbanksoftware wird LibreOffice Base verwendet (kostenlos verfügbar).
    \item Zusätzlich zur Arbeit im Informatik-Unterricht wird auch Arbeit zuhause notwendig sein. Arbeitsteilung ist erwünscht!
    \item Festhalten von TODOs und Bearbeitungsstand auf dem Project-Board in Mebis.
    \item Alle Abgaben via Mebis im PDF oder ODB (Libreoffice Datenbank) Format.
\end{itemize}

\section{Inhaltliche Anforderungen}
\begin{itemize}
    \item Datenbank mit mindestens 4 inhaltlichen Tabellen (reine Beziehungstabellen zusätzlich)
    \item alle 3 Kardinalitäts-Typen
    \item Erfüllung der 3 Normalformen
    \item Bedienung der Datenbank ausschließlich über SQL-Abfragen, Formulare und (optional) Berichte.
\end{itemize}

\section{Grober Zeitplan}

\begin{enumerate}
    \item \textbf{KW 49 (ab 04.12.):} Anforderungsanalyse\\
       $\rightarrow$  Abgabe: Lastenheft \textbf{bis 15.12., 23:59 Uhr}
    \item \textbf{KW 49 (ab 11.12.):} Datenbankentwurf\\
       $\rightarrow$  Abgabe: Pflichtenheft \textbf{bis 12.01., 23:59 Uhr}
    \item \textbf{KW 50 (ab 08.01.):} Umsetzung der Datenbank
    \item \textbf{KW 02 (ab 15.01.):} Umsetzung der Datenbank\\
       $\rightarrow$  Abgabe: 1Zwischenstand Datenbank \textbf{bis 26.01., 23:59 Uhr}
    \item \textbf{KW 03 (ab 22.01):} Tests + Dokumentation\\
       $\rightarrow$  Abgabe: Test-Protokoll \textbf{bis 02.02., 23:59 Uhr}
    \item \textbf{KW 04 (ab 29.01.):} Abschluss der Dokumentation und Reflexion\\
       $\rightarrow$  Abgabe: Finale Datenbank+Dokumentation+Reflexion \textbf{bis 09.02., 23:59 Uhr}
\end{enumerate}

\section{Anforderung einzelner Phasen}

\subsection{Anforderungsanalyse}
\begin{itemize}
    \item Allgemeine Beschreibung des Produkts und Einsatz (Lastenheft)
    \item mindestens 5 User Stories (Lastenheft)
    \item funktionale und nicht-funktionale Anforderungen (Lastenheft)
    \item Eine Nummerierung (o.ä.) der einzelnen Elemente, um später darauf verweisen zu können empfiehlt sich.
    \item Formulierung einzelner TODOs (Project Board)
    \item erste Gedanken zur Umsetzung und Machbarkeit
\end{itemize}

\subsection{Datenbankentwurf}
\begin{itemize}
    \item Entwurf des Datenbankschemas\\
    $\rightarrow$ Festhalten mittels Klassendiagramm (Pflichtenheft)
    \item Entwurf User Interface (Abfragen, Formulare)
    $\rightarrow$ Festhalten mittels Skizzen/Beschreibung (Pflichtenheft)\\
    \item Zuordnung der Inhalt zu den Anforderung aus dem Lastenheft
\end{itemize}

\subsection{Umsetzung}
\begin{itemize}
    \item Umsetzung des Plans aus dem Pflichtenheft
    \item Haltet euch möglichst genau an euer Pflichtenheft. Notfalls müsst ihr Änderungen vornehmen.
    \item Bearbeitung möglichst arbeitsteilig
\end{itemize}

\subsection{Tests}
\begin{itemize}
    \item kontinuierliches Testen während der Umsetzung
    \item Überprüfung des Endprodukts zuerst anhand des Pflichtenhefts...
    \item ...anschließend anhand des Lastenhefts
    \item Gesamtbeurteilung, ob das Projekt erfolgreich umgesetzt wurde.
    \item Festhalten aller Ergebnisse in einem Test-Protokoll 
\end{itemize}

\subsection{Dokumentation}
\begin{itemize}
    \item Lastenheft, Pflichtenheft, Testprotokoll
    \item Bedienungsanleitung
    \item Finales Datenbankschema
    \item Alle SQL-Abfragen
    \item Optisch ansprechendes Dokument!
\end{itemize}

\subsection{Bericht/Reflexion über...}
\begin{itemize}
    \item ...die einzelnen Projektphasen und aufgetretene Schwierigkeiten.
    \item ...während der Umsetzung erfolgte Änderungen gegenüber Pflichtenheft.
    \item ...was ihr wieder so/anders machen würdet.
    \item ...die erfolgte Arbeitsteilung.
\end{itemize}

\section{Benotung}
\subsection{Note 1: Produkt und Arbeitsphase}
\begin{itemize}
    \item Funktionalität der Datenbank, Erfüllung der Spezifikation
    \item Umsetzung des Wasserfallmodells
    \item Beobachtung während Arbeitsphase
\end{itemize}

\subsection{Note 2: Dokumentation und Reflexion}
\begin{itemize}
    \item Verständlichkeit, Sprache
    \item inhaltliche und sachliche Übereinstimmung mit dem Produkt 
    \item differenzierte Reflexion der eigenen Arbeit
\end{itemize}